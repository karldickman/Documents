% LaTeX file for resume 
% This file uses the resume document class (res.cls)

\documentclass{res} 
%\usepackage{helvetica} % uses helvetica postscript font (download helvetica.sty)
%\usepackage{newcent}   % uses new century schoolbook postscript font 
\setlength{\textheight}{9.5in} % increase text height to fit on 1-page 

\newcommand{\employment}[6]{\vspace{-0.1in}\begin{tabbing}\hspace{2.3in}\= \hspace{2.6in}\= \kill
    \textbf{#1} \>#2     \>#3\\
                             \end{tabbing}\vspace{-20pt}
                             Supervisor: #4\\
                            Main Phone: #5\\
                             #6}

\newcommand{\reference}[6]{\textbf{#1}\\#2\\Phone Number: #3\\Email Address:
\texttt{#4}\\Professional Relationship: #5\\#6}

\begin{document} 

\name{KARL S. DICKMAN\\[12pt]}     % the \\[12pt] adds a blank
				        % line after name      

\address{\bf  PRESENT ADDRESS\\0615 SW Palatine Hill Rd.\\Portland, OR 97219\\(503) 887-4920}
\address{\bf PERMANENT ADDRESS \\ 556 NW Trenton Avenue \\Bend, OR 97701 \\  (541) 312-8219}
                                  
\begin{resume}

\section{Job Objective}
    I am searching for a position that will strenuously challenge my programming
    skills and encourage the growth of those skills.
 
\section{Education}          
    \textbf{Lewis \& Clark College} \\
    Bachelor of Computer Science and Mathematics, May 2010 \\
    Degree-related GPA 3.8 \\
    Cumulative GPA 3.6 \\
 
\section{Employment Experience}
    \employment{Student Worker}{Bon Appetit}{Sep. 2008 to
    May 2010}{Landis Jurd}{(503) 768-7890}{Customer service at Lewis \& Clark
    dining hall, including serving food, working the till, and cleaning the
    service area.}

    \employment{Reception Desk Attendant}{Lewis \& Clark College}{Jan. 2010 to
    Mar. 2010}{Mark Minty}{(503) 768-7546}{Greet customers at the reception
    desk, answer phones, check out towels and other athletic equipment.}

    \employment{Equipment Cage Attendant}{Lewis \& Clark College}{Sep. 2008
    to Dec. 2009}{Mark Minty}{(503) 768-7546}{Laundry service and equipment
    checkout for Lewis \& Clark Athletic Department.}

    \employment{Painter}{Lewis \& Clark College}{May 2008 to Aug.
    2009}{Richard Austin}{(503) 768-7864}{Summer painting and repair work on
    dormitories and other college property.}

    \employment{Intern}{R\&W Engineering (Bend, Oregon)}{Jan. 2009}{Anthony
    Dickman}{(541) 322-8940}{CAD drafting.  Wrote Visual Basic macros for Excel to calculate
    fluid flow through pipes of arbitrary shape.}

    \employment{Data Entry}{Lewis \& Clark College}{Sep. 2007 to Feb.
    2008}{Dr. David Fix}{(503) 768-7068}{Recorded Track \& Field marks for High
    Schoolers in Oregon, Washington, and California.}

    \employment{Intern}{R\&W Engineering (Bend, Oregon)}{May 2007 to Aug. 2007}{Anthony
    Dickman}{(541) 322-8940}{Redrafted CAD details.}

\section{Professional References}
    \reference{Richard Austin}
    {Lead Painter, Lewis \& Clark College}
    {(503) 768-7181}
    {raustin@lclark.edu}
    {May 2008 to August 2009}
    {As the Lead Painter, Richard is directly in charge of supervising the
    student workers when the dorms are repainted every summer.}

    \reference{Dr. Peter Drake}
    {Associate Professor of Computer Science, Lewis \& Clark College}
    {(503) 768-7539}
    {drake@lclark.edu}
    {September 2008 to May 2010}
    {Dr. Drake teaches Data Structures and Algorithms, Theory of Computation,
    Artificial Intelligence, and Software Development.}
 
    \reference{Dr. Jeff Ely}
    {Associate Professor of Computer Science, Lewis \& Clark College}
    {(503) 768-7561}
    {ely@lclark.edu}
    {September 2006 to May 2010}
    {Dr. Ely teaches Computer Graphics and Advanced Computer Graphics.  He is my
    academic advisor.}

    \reference{Landis Jurd}
    {Dining Room Manager, Bon Appetit}
    {(503) 768-7895}
    {ljurd@lclark.edu}
    {September 2008 to May 2010}
    {Mr. Jurd is directly in charge of the student workers in the dining hall.}

	\reference{Dr. Jens Mache}
    {Associate Professor of Computer Science, Lewis \& Clark College}
    {(503) 768-7564}
    {jmache@lclark.edu}
    {January 2009 to December 2009}
    {Dr. Mache teaches Computer Architecture and Assembly Language, as well as
    Internet Security.}

\section{Awards and Achievements}
    \textbf{Joe Huston Male Athlete of the Year}\\
    Awarded by Lewis \& Clark College in April 2010 for athletic achievement.

    \textbf{Bronze Medal, Dell Smith Scholar-Athlete of the Year}\\
    Awarded by Lewis \& Clark College in April 2010 for academic and athletic
    achievement.

    \textbf{All-Academic Cross-Country Team}\\
    Awarded by the United States Track and Field and Cross-Country Coaches
    Association in December 2009 for academic and athletic achievement.

    \textbf{All-Academic Cross-Country Team}\\
    Awarded by the Northwest Conference in June 2009 for academic achievement.
 
    \textbf{All-Academic Track \& Field Team}\\
    Awarded by the Northwest Conference in June 2009 for academic achievement.
 
    \textbf{Dean's List}\\
    Awarded by Lewis \& Clark College in May 2009 for academic achievement.
 
    \textbf{Pioneer All-Academic First Team}\\
    Awarded by Lewis \& Clark College in April 2009 for academic achievement.
 
    \textbf{Dean's List}\\
    Awarded by Lewis \& Clark College in December 2007 for academic achievement.

\section{Skills}          
    \textbf{Computer Operating Systems}
        \begin{itemize}
 	        \item Linux (Proficient)
            \item Mac OS X (Proficient)
            \item Windows (Proficient)
        \end{itemize}
    \textbf{Computer Programming Languages}
        \begin{itemize}
            \item Bash (Proficient)
            \item C (Proficient)
            \item HTML (Proficient)
 	        \item Java (Advanced)
            \item Javascript (Proficient)
            \item PHP (Proficient)
 	        \item Python (Proficient)
            \item XML (Proficient)
        \end{itemize}
    \textbf{Database Management Systems}
        \begin{itemize}
 	        \item MySQL (Proficient)
        \end{itemize}
    \textbf{Development Tools}
        \begin{itemize}
            \item Eclipse (Proficient)
        \end{itemize}
    \textbf{Engineering}
        \begin{itemize}
 	        \item Autodesk AutoCAD (Proficient)
        \end{itemize}
    \textbf{Foreign Languages}
        \begin{itemize}
 	        \item German (Read)
        \end{itemize}
    \textbf{Office}
        \begin{itemize}
            \item Microsoft Excel (Proficient)
 	        \item Microsoft Word (Proficient)
        \end{itemize}
    \textbf{Publishing}
        \begin{itemize}
 	        \item \LaTeX{} (Proficient)
        \end{itemize}
    \textbf{Version Control Systems}
        \begin{itemize}
            \item git (Proficient)
            \item CVS (Proficient)
        \end{itemize}

\section{Coursework}
    \textbf{Advanced Computer Graphics}\\
    Lewis \& Clark College\\
    Advanced three-dimensional computer graphics. Z-buffer algorithms, Phong
    smooth shading, ray tracing, texture mapping and spline patches.

    \textbf{Algorithm Design and Analysis (A)}\\
    Lewis \& Clark College\\
    Introduction to the design and analysis of algorithms.  Balanced binary
    search trees; bit vector; hash tables; heaps; dynamic programming;
    algorithms including incremental, divide and conquer, greedy, graph.

    \textbf{Artificial Intelligence (A)}\\
    Lewis \& Clark College\\
    Design and construction of intelligent computer systems. Agents and
    environments; blind and informed search; heuristics; game play, minimax, and
    alpha-beta pruning; robotics; machine learning; philosophical issues
    including definitions of intelligence.

    \textbf{Combinatorics}\\
    Lewis \& Clark College\\
    Introduction to combinatorial theory, including one or more of the
    following: enumeration, algebraic enumeration, optimization, graph theory,
    coding theory, design theory, finite geometries, Latin squares, posets,
    lattices, Polya counting, Ramsey theory.

    \textbf{Computer Architecture and Assembly Language (A)}\\
    Lewis \& Clark College\\
    Computer design concepts and assembly languages. Topics chosen from the
    following: digital logic; arithmetic/logic unit design; bus structures; VLSI
    implementation; SIMD, MIMD, and RISC architectures; instruction sets; memory
    addressing modes; parameter passing; macro facilities.

    \textbf{Cryptography (B+)}\\
    Lewis \& Clark College\\
    Classical cryptography including Vigenere, double-columnar transposition,
    and Enigma. Cryptanalytic techniques for those systems.  Computer
    cryptography including DES, RSA, El Gamal, the Knapsack Cipher, and
    cryptanalysis of those systems.

    \textbf{Discrete Mathematics (A--)}\\
    Lewis \& Clark College\\
    Basic techniques of abstract formal reasoning and representation used in the
    mathematical sciences. First order logic, elementary set theory, proof by
    induction and other techniques, enumeration, relations and functions,
    graphs, recurrence relations.

    \textbf{Electricity and Magnetism (A)}\\
    Lewis \& Clark College\\
    Introduction to electricity, magnetism, and their interactions. Electric
    fields and electric potentials. Phenomena of capacitance, currents,
    circuits. Forces on moving charges described in terms of the magnetic field.
    Effects of time-varying electric and magnetic fields, in both vacuum and
    matter: induction, alternating current circuits, electromagnetic waves.

    \textbf{Internet Security (A--)}\\
    Lewis \& Clark College\\
    Buffer overflows, network sniffing, attacks on web applications.

    \textbf{Software Development}\\
    Lewis \& Clark College\\
    Development of large software systems by teams of programmers.  Problem
    specification, system design, testing, software frameworks, design patterns.

    \textbf{Theory of Computation (A)}\\
    Lewis \& Clark College\\
    Chomsky hierarchy, computability theory (including the halting problem and
    other impossible problems), complexity theory (including P vs.  NP),
    reducibility.
 
\section{Certifications}
    \begin{itemize}
        \item Food Handler (September 2008)
    \end{itemize}

\section{Extracurricular Activities}          
    \begin{itemize}
 	    \item
            \textbf{Lewis \& Clark Cross-Country Team}\\
            Actively Participated: 80 hours per month\\
            Dates of Involvement: September 2006 to November 2009\\
            Achievements:
            \begin{itemize}
                \item Team Captain, 2009
                \item National qualifier, 2009
                \item West Region Second Team, 2009
                \item Northwest Conference First Team, 2009
                \item West Region Fifth Team, 2008
            \end{itemize}
 	    \item
            \textbf{Lewis \& Clark Track and Field Team}\\
            Actively Participated: 80 hours per month\\
            Dates of Involvement: January 2007 to April 2010
        \item
            \textbf{Student Athletic Advisory Committee}\\
            Description: SAAC is the NCAA-mandated liaison between student-athletes
            and the other groups and organizations of the sponsoring college.\\
            Actively Participated: 4 hours per month\\
            Dates of Involvement: September 2009 to May 2010
    \end{itemize}
 
\end{resume}
\end{document}
